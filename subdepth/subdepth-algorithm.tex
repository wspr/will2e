\documentclass{article}
\usepackage{calc,color,graphicx,url}
\def\cs#1{\texttt{\char`\\#1}}
\begin{document}

\section{Algorithm}

\makeatletter

First, some typical examples for context:
\def\YPOS{%
  \pdfsavepos\write-1{\the\pdflastypos}%
}
\begin{center}
    $A^{A}_{B}$\quad
    $A^{x}_{y}$\quad
    $A^{\frac{x}{y}}_{2}$\quad
    $A_2$\quad
    $C_2H_5^+$\quad
    $A_{2\YPOS}A^1_{2\YPOS}A^g_{2\YPOS}A^g$
\end{center}


\newcommand\HDrule[2]{%
  \hbox{%
    {\color{red}\vrule height \dimexpr(#1) depth 0pt width 1ex\relax}%
    \kern-1ex\relax
    {\color{black}\vrule height 0pt depth \dimexpr(#2) width 1ex\relax}%
  }%
}

\def\supsym#1{%
  \smash{\color{red}%
    \makebox[0pt][l]{\rule[-0.4pt]{0.7cm}{0.4pt}}
    \rule
      [-\the\fontdimen#1\textfont2]
      {0.4pt}{\the\fontdimen#1\textfont2}}
   i}
\def\subsym#1{%
  \smash{\color{red}%
    \makebox[0pt][l]{\rule[-0.4pt]{0.7cm}{0.4pt}}
    \rule{0.4pt}{\the\fontdimen#1\textfont2}}
   i}

\savebox\@tempboxa{$~$}%
\edef\reset{%
   \fontdimen13\noexpand\textfont2=\the\fontdimen13\textfont2
   \fontdimen14\noexpand\textfont2=\the\fontdimen14\textfont2
   \fontdimen15\noexpand\textfont2=\the\fontdimen15\textfont2
   \fontdimen16\noexpand\textfont2=\the\fontdimen16\textfont2
   \fontdimen17\noexpand\textfont2=\the\fontdimen17\textfont2
 }

\def\a#1{\noindent
{\footnotesize\ttfamily1.5\cs{fontdimen#1}:}
\hfill
\reset$\displaystyle\sum A^{\supsym{13}}$\quad
\fontdimen#1\textfont2=1.5\fontdimen#1\textfont2
$\displaystyle A^{\supsym{13}}$\hfill
\reset$\sum A^{\supsym{14}}$\quad
\fontdimen#1\textfont2=1.5\fontdimen#1\textfont2
$A^{\supsym{14}}$\hfill
\reset$\sqrt{A^{\supsym{15}}}$\quad
\fontdimen#1\textfont2=1.5\fontdimen#1\textfont2
$\sqrt{A^{\supsym{15}}}$\hfill
\reset\null\par\bigskip}

\def\b#1{\noindent
{\footnotesize\ttfamily2\cs{fontdimen#1}:}\hfill
\reset$A_{\subsym{16}}$\quad
\fontdimen#1\textfont2=2\fontdimen#1\textfont2
$A_{\subsym{16}}$\hfill
\reset$A^B_{\subsym{17}}$\quad
\fontdimen#1\textfont2=2\fontdimen#1\textfont2
$A^B_{\subsym{17}}$\hfill
\reset\null\par\bigskip}

In the examples to follow the height and depth of the boxes is important and indicated with a stacked box as so: `\HDrule{2ex}{1ex}'. (That example shows a height of 2ex and a depth of 1ex.)

The intricacies of \TeX's behaviour for maths typesetting are well described by Bogusław `Jacko' Jackowski in his TUGboat paper `Appendix G illuminated':
\url{https://www.tug.org/TUGboat/tb27-1/tb86jackowski.pdf}

Superscript symbols are placed vertically according to fontdimens 13 (display), 14 (text), and~15 (cramped).

\bigskip
\a{13}
\a{14}
\a{15}

For subscripts, fontdimens 16 and 17 are used when and when not a superscript is also present, respectively.

\bigskip
\b{16}
\b{17}

Fontdimens 18 and 19 are used to place the subscript and superscript, respectively, when they are attached to a boxed formula. But we don't care for that special case, because it already looks fine.

Furthermore, if subscripts and superscripts clash then their spacing will be adjusted to compensate.

First of all, the superscript must not extend too low, or it is raised to not exceed this minimum:
\begin{center}
\@for\ii:={0pt,1pt,2pt,3pt,4pt,5pt}\do{%
  \scalebox{2}{$A^{\HDrule{1ex+\ii}{0.5ex+\ii}}$\quad}}
\end{center}
Similarly, the subscript cannot extend too high:
\begin{center}
\@for\ii:={0pt,1pt,2pt,3pt,4pt,5pt}\do{%
  \scalebox{2}{$A_{\HDrule{1ex+\ii}{0.5ex+\ii}}$\quad}}
\end{center}
Note that the `too low' position for the superscript and the `too high' for the subscript overlap. There are further restrictions when they clash. When the superscript clashes with the subscript:
\begin{center}
\@for\ii:={0pt,1pt,2pt,3pt,4pt,5pt}\do{%
  \scalebox{2}{$A_{\HDrule{1ex}{0.5ex}}^{\HDrule{1ex+\ii}{0.5ex+\ii}}$\quad}}
\end{center}
And when the subscript clashes with the superscript:
\begin{center}
\@for\ii:={0pt,1pt,2pt,3pt,4pt,5pt}\do{%
  \scalebox{2}{$A_{\HDrule{1ex+\ii}{0.5ex+\ii}}^{\HDrule{1ex}{0.5ex}}$\quad}}
\end{center}
And when they both grow:
\begin{center}
\@for\ii:={0pt,1pt,2pt,3pt,4pt,5pt}\do{%
  \scalebox{2}{$A_{\HDrule{1ex+\ii}{0.5ex+\ii}}^{\HDrule{1ex+\ii}{0.5ex+\ii}}$\quad}}
\end{center}

The parameter that mainly controls the details of all these adjustments is the x-height of the font, fontdimen 5.

What this package attempts to do is unify the subscript placement with and without a superscript. By default the spacing is as so:

\begin{center}
  \scalebox{2}{%
    $A_{\HDrule{1ex}{0.25ex}}$\quad
    $A^{\HDrule{1ex}{0.25ex}}_{\HDrule{1ex}{0.25ex}}$\quad
  }
\end{center}



\makeatother

\end{document}
